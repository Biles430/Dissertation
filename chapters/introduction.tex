\chapter{Introduction}

Boundary layers form when a fluid flows tangentially to a no-slip surface boundary (e.g. a solid wall) \cite{Prandtl1905}.  If a temperature difference exists between the no-slip surface boundary and the fluid, a thermal boundary layer will form as well \cite{Kays1980}.  By definition, boundary layers are very thin compared to a characteristic geometric dimension of the no-slip surface (e.g., the length of a solid wall over which the fluid flows). In spite of its relative thinness, the dynamics internal to the boundary layer determine the rate at which mass, momentum, and energy are transferred between the fluid and the surface boundary. In the majority of engineering systems as well as geophysical flows, the dynamical flow state of the boundary layer is turbulent.  It follows that the need to reliably analyze, predict, and control the transport of mass, momentum, and energy in turbulent boundary layers is critically important across a broad spectrum of technological applications and scientific disciplines \cite{Fox2012}. Owing to this importance, there has been extensive--and continuing--research to better understand the underlying boundary layer transport mechanisms at work \cite{Clauser1956}. The extensive body of research includes experimental, theoretical, and numerical studies, where the overwhelming majority of these studies have focused on laboratory-scale canonical wall-bounded flows such as fully-developed pipe and channel flow or zero pressure gradient (ZPG) boundary layer flow. It is fair to say that much has been learned about the dynamics of these canonical flows \cite{Sreenivasan1997}, although there is much still to learn \cite{Sreenivasan1999, Jimenez2012}. Significantly less, however, is known about flows that fall outside the domain of these well-studied flows. 

The broad objective of this dissertation is to better understand (a) the boundary layer structure at high Reynolds numbers and (b) heat transfer in non-equilibrium boundary layers.  Although the experimental investigations of (a) and (b) can be considered independent as they are conducted in different flow facilities and with different initial and boundary conditions, the unifying theme of the dissertation is to expand the knowledge base of turbulent boundary layer dynamics beyond canonical laboratory-scale flows. Moreover, it is important to study the types of flow investigated in this dissertation for practical purposes since many natural and engineering fluid flows are at high Reynolds number (e.g., atmospheric flows and flow around large-scale aviation and marine vessels). Similarly, heat transfer in non-equilibrium wall bounded flow plays a critical role in the performance, efficiency, and life-cycle of many engineered systems (e.g., piston engine, gas turbines).
%%%%%%%%%%%%%%%%%%%%%%%%%%%%%%%%%%%%%%%%%%%
\section{Boundary Layers}

%%%%%%%%%%%%%%%%%%%%%%%%%%%%%%%%%%%%%%%%%%%
\section{Thermal Boundary Layers}


%%%%%%%%%%%%%%%%%%%%%%%%%%%%%%%%%%%%%%%%%%%
\section{Non-Equilibrium Boundary Layers}

%%%%%%%%%%%%%%%%%%%%%%%%%%%%%%%%%%%%%%%%%%%
\section{Experimental Investigations}
%%%%%%%%%%%%%%%%%%%%%%%%%%%%%%%%%%%%%%%%%%%%%%%%%
%\section{High Reynolds number boundary layer studies}
%Turbulence is intrinsically a high-Reynolds-number phenomenon, yet studies of high Reynolds number boundary layer flows are few in number, and of these studies, those obtained under controlled conditions with sufficient spatial measurement resolution to resolve the near-wall region are relatively rare. The small number of physics quality high Reynolds number turbulent boundary layer facilities in the world is a testament to the difficulties. Yet, despite limited experimental facilities, it is important to study high $Re$ flow to test hypotheses and predictions of turbulence theories (e.g., K41) and scaling laws (e.g. the logarithmic law of the wall) and to evaluate the robustness of reduced-order models of turbulent flows. The reason being that turbulence theory, scaling laws, and reduced models are typically derived in the infinite $Re$ limit. Moreover, most natural flow phenomena occur on large scales hence large Reynolds numbers. In this respect high Reynolds number data is needed to validate models for forecasting weather, ocean temperatures and currents, and the ecological and environmental effects of strong singular events (i.e., El N\~{i}no, volcanic explosions, and the like), containment and/or early warning of ocean spillage and airborne pollutants, minimizing catastrophic effects of atmospheric turbulence on aircraft, and so forth.
%
%In this dissertation high Reynolds number boundary layer flows are produced and studied in the Flow Physics Facility (FPF) at the University of New Hampshire. The FPF employees the `big and slow' approach to generate thick (order 1m) boundary layers at high $Re$. The benefit of a thick boundary layer is that the near-wall region can be probed with sufficient spatial resolution to resolve the dynamically important near-wall flow structures. The specific objective of my ongoing and future experimental studies in the FPF is to measure and quantify the flow structure of the boundary layer in the so-called inertial layer. This will be achieved using particle image velocimetry (PIV) to acquire planar fields of velocity in the streamwise wall-normal plane. The PIV data will be used to characterize two specific structures: uniform momentum zones (UMZ) and vortical fissure (VF)  \cite{Adrian2000}.
%
%Characterization of UMZs and VFs has previously been performed by \cite{Silva2017} using thresholding strategies applied to PIV data. Here they they found that the thickness of the VF remains consistent with $Re$ and is on the order of the Taylor microscale and that the jump $\Delta U$ across the VF is likely to become $Re$ invariant when scaled by $u_\tau$. In the present study, thresholding strategies similar to \cite{Silva2017} will first be used to identify and characterize the UMZ and VF structures. (We will also investigate other strategies besides thresholding to identify VFs and UMZs). Once identified and isolated, the PIV data will be analyzed to investigate the internal dynamics of VFs and UMZs.  The impact of this work is two-fold (1) to provide additional data that can be used to corroborate or refine the findings of \cite{Silva2017} and (2) to provide a first assessment and analysis of the internal structure of a VF. 

%%%%%%%%%%%%%%
%%FROM PAPER%%
%%%%%%%%%%%%%%

A momentum boundary layer forms when a fluid flows through or over a body that has a no-slip surface boundary (e.g. a solid wall) \cite{prandtl1908}. If a temperature difference exists between the no-slip surface boundary and the fluid, a thermal boundary layer will form as well \cite{Kays1980}. The size of the thermal boundary layer relative to the momentum boundary layer will depend on the Prandtl number, $Pr = \nu/\alpha$, where $\nu$ and $\alpha$ are the kinematic viscosity and thermal diffusivity of the fluid, respectively. In spite of its relative thinness compared to the size of the body, the dynamics of the boundary layer determine the rate at which mass, momentum, and energy are transferred between the fluid and the surface boundary. In the majority of engineering systems as well as geophysical flows, the dynamical flow state of the boundary layer is turbulent.  It follows that the need to reliably analyze, predict, and control the transport of mass, momentum, and energy in turbulent boundary layers is critically important across a broad spectrum of technological applications and scientific disciplines \cite{Fox2012}. Owing to this importance, there has been extensive and continuing experimental, theoretical and numerical research to better understand the underlying transport mechanisms at work \cite{Clauser1956}. The extensive body of research includes experimental, theoretical, and numerical studies, where the overwhelming majority of these studies have focused on so-called canonical wall-bounded flows such as fully-developed pipe and channel flow or ZPG boundary layer flow. It is fair to say that much has been learned about the dynamics of these canonical flows \cite{Sreenivasan1997}, although there is much still to learn \cite{Sreenivasan1999, Jimenez2012}. Significantly less, however, is known about the dynamics of non-equilibrium boundary layer flows. The need to address this knowledge gap is important since many application relevant and geophysical flows exhibit non-equilibrium boundary layer behaviors \cite{Hara2009}.

In canonical equilibrium boundary layer flows, the time scales over which the mean field vary are large compared to local turbulent time scales. The turbulent field rapidly adjusts to mean field variations, and the flow exhibits universal behaviors when scaled by local parameters \cite{townsend1976}. One very important universal behavior is the logarithmic dependence of the mean velocity profile $\overline{u}$ in the so-called logarithmic region,

\begin{equation}
{\overline u^+}=\frac{1}{\kappa}\log(y^+)+C_1, \label{eq:lawofwall}
\end{equation}

\noindent  where the superscript $^+$ denotes normalization by the friction velocity $u_\tau = \sqrt{\tau_w / \rho}$ and kinematic viscosity $\nu$, where $\tau_w$ is the shear stress at the wall and $\rho$ the fluid density; $1/\kappa$ (typically $\kappa$ is called the von K\'{a}rm\'{a}n coefficient) is the slope; $y^+$ is the wall-normal coordinate; and $C_1$ is the intercept at $y^+ = 1$. Equation~\ref{eq:lawofwall} is referred to as the law of the wall with coefficients (at sufficiently high Reynolds number) $\kappa\approx0.4$ and $C_1\approx5$, varying for a given canonical flow type \cite{Nagib2008}. 

The temperature distribution in the boundary layer, using similar dimensional scaling arguments used to derive Eq.~\ref{eq:lawofwall}, yields the law of the wall for the mean temperature distribution,
\begin{equation}
{\overline{\Theta^+ }} =  \frac{\overline{T_w - T(y)}}{T_\tau} =\frac{1}{\kappa_T}\log(y^+)+C_2(Pr), \label{eq:lawofwallT}
\end{equation}

\noindent \noindent  where $T_w$ is the wall temperature, $T(y)$ is the temperature in the boundary layer, the superscript $^+$ denotes normalization by the friction temperature $T_\tau = q''_w / (\rho c_p u_{\tau} )$, where $q''_w$ is wall heat  flux, $c_p$ is specific heat and $u_\tau$ is the friction velocity. The dimensionless coefficients $1/\kappa_T$ is the slope and $C_2$, which is a function of $Pr$, is the intercept at $y^+ = 1$. In general at sufficiently high Reynolds number, $\kappa_T \approx 0.48$ but also varies for a given canonical flow type \cite{kader1972}.

In non-equilibrium boundary layers, the time scales over which the mean field vary are comparable (or smaller) to local turbulent time scales, and the flow field cannot be characterized solely in terms of local parameters \cite{townsend1976}. Such rapid changes in the mean momentum field typically result from pressure gradients, wall curvature, strong three-dimensionality, wall roughness, or dynamic walls. For non-equilibrium boundary layers in which an equilibrium boundary layer flow experiences a localized perturbation (e.g., flow over an obstacle/cavity or flow subjected to a pressure gradient), there has been considerable, and continuing, research to understand the redistribution of the momentum field when the equilibrium state is disturbed \cite{Antonia1977, Bradshaw1972a, Bandyopadhyay1993, Castro1998}. In general, these studies show that: (a) in a small local region near the perturbation the log-layer is obliterated (i.e., the law of the wall given by Eq.~\ref{eq:lawofwall} does not hold), (b) downstream of the perturbation, in the so-called recovery region, an internal stress equilibrium layer grows and the boundary layer recovers towards equilibrium. Conceptually, the effect of (b) relative to Eq.~\ref{eq:lawofwall} is a spatially developing slope and intercept that approach their universal values at the edge of the recovery region; the functional form of the spatial dependence depending on the perturbation. 

While the effects of non-equilibrium boundary layers on the velocity field have been considerably studied, heat transfer in non-equilibrium boundary layers has received far less attention. Nevertheless, despite somewhat limited data, it is a well-accepted fact that the law of the wall for temperature is more affected by mean field variations than the velocity field \cite{Blackwell_1972, Kader1991, Bradshaw1995, Kong2001, Houra2006, Wang2008}. For example, in non-equilibrium boundary layer flow subjected to a pressure gradient, the constants in Eq.~\ref{eq:lawofwallT} vary significantly with pressure gradient while the constants in Eq.~\ref{eq:lawofwall} vary little. This difference in sensitivity is unexpected given that Eqs.~\ref{eq:lawofwall}-\ref{eq:lawofwallT} were derived from analogous dimensional scaling arguments, and has brought into question the validity of the law of the wall \cite{Bradshaw1995}. Moreover, the high sensitivity of the temperature field to pressure gradient flows is remarkable since the pressure gradient does not appear in the transport equation for temperature. 
The consensus, although not entirely well-understood, is that while the law of the wall for velocity is fairly resilient, the law of the wall for temperature is very strongly affected by upstream disturbances. The implication is that the scaling used to derive the law of the wall fails to describe the behaviors of the mean dynamics (especially for temperature) when there are large gradients in the mean flow direction. Consequently, to capture non-equilibrium effects on mean field dynamics, the present state of the research is to reformulate the law of the wall through the use of new scaling laws \cite{Durbin1992, George1993, Cruz1998, Cruz2002, Wang2008} or use single-point closure models (i.e., eddy viscosity or mixing length models) informed by experimental or numerical data \cite{Cebeci1988}.

Extrapolating the results discussed above to \emph{strong} non-equilibrium flows, in which mean field perturbations vary rapidly in magnitude both spatially and temporally (e.g., in-cylinder engine flows or other reciprocating machinery), a logical conclusion with respect to the law of the wall is that Eqs.~\ref{eq:lawofwall}-\ref{eq:lawofwallT} will either (a) not hold or (b) the slope and intercept will vary strongly in space and time. Moreover, the high sensitivity of the temperature field to mean field perturbations strongly suggests that computational fluid dynamic (CFD) simulations utilizing wall functions based on equilibrium boundary layer behaviors will not accurately capture heat transfer in strong non-equilibrium flows or perhaps even fail spectacularly \cite{Launder1988}. 

The potential for CFD simulations to accurately predict boundary layer transport depends on the specifics of the closure model and wall functions used. In general, the development of turbulent closure models and wall functions are informed, refined, and validated by experimental data. One obstacle for formulating new engineering heat transfer models that better captures the physics of non-equilibrium flows (i.e., flows with complex dynamics) is the lack of robust experimental data needed to both develop and validate models.

The lack of experimental studies is not surprising given that controlling thermal boundary conditions is non-trivial, and the simultaneous measurement of temperature and velocity fluctuations in turbulent boundary layers with heat transfer is very difficult. In addition, direct measurement of the wall-heat flux, which is the primary scaling variable to study thermal boundary layers, is challenging. 
The aim of the present work is to develop a unique thermal wall plate that can be used to control the thermal boundary conditions in non-equilibrium boundary layer flows. 
Owing to the strong spatial variations of the flow associated with non-equilibrium boundary layers, this is a nontrivial but important objective needed to better understand thermal transport in these types of flows. 
The facility described here, the Non-Equilibrium And Thermal (NEAT) boundary layer wind tunnel, will be used to advance the fundamental knowledge base of non-equilibrium boundary layer transport as well as to develop and validate Reynolds-Averaged-Naiver Stokes (RANS) turbulence models for non-equilibrium thermal boundary layer flows.   
For the latter use, the ability to modify the thermal conditions in a variety of ways is beneficial so that RANS models can be tested and validated in uniquely different flows but in the same flow facility using the same measurement techniques. The need for the facility and corresponding measurements is evident by the fact that the experimental measurement of Perry \textit{et al.}\cite{Perry1966} and Blackwell \textit{et al.}\cite{Blackwell_1972} remain the primary datasets utilized in the validation of DNS of thermal boundary layers \cite{Araya2012}
