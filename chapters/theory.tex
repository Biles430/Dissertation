\chapter{Theory}

%%%%%%%%%%%%%%%%%%%%%%%%%%
\section{Boundary Layers}
-detail general equations for boundary layer flow and describe assumptions made to arrive at equations and empirical formulations based off of equation.
Boundary layer thickness, \\

%%%%%%%%%%%%%%%%%%%%%%%%%%
\section{Reynolds Analogy}
-Detail how BL work is imposed onto the velocity field and show the similarities between the two field transport equations.
Discuss Reynolds analogy, \\
\nomenclature{$St_T$}{Stanton number for an incompressible boundary layer flow over an isothermal plate}

Here 

\begin{equation}
\Theta = \frac{T_w - T(y)}{T_w - T_\infty},
\end{equation}

\noindent where $T_\infty$ denotes the freestream temperature. The inner normalization is described in Eq.~\ref{eq:lawofwallT}. The value of the wall heat flux needed for the inner-normalization was approximated using Colburns formulation given by 

\begin{equation}
St_T = \frac{q''_w}{\rho c_p U_\infty (T_{w} - T_\infty)} = Pr^{-2/3} \left(\frac{u_\tau}{U_\infty}\right)^2 
%q_w = \rho c_p U_\infty Pr^{-2/3} (T_{wall} - T_\infty)\left(\frac{u_\tau}{U_\infty}\right)^2 
\label{eq:qw}
\end{equation}

\noindent where 

%%%%%%%%%%%%%%%%%%%%%%%%%%%%%%
\section{Non-equilibrium Flow}
-Discuss how momentum fields depart from equil when non-equil behavior is introduced.
Discuss thermal field, break down of Reynolds analogy\\

%%%%%%%%%%%%%%%%%%%%%%%%%%%%%%
\section{Determination of Wall Heat Flux}



In canonical equilibrium turbulent boundary layer flow, the time scales over which the mean field vary are large compared to local turbulent time scales. The turbulent field rapidly adjusts to mean field variations, and the flow exhibits universal behaviors when scaled by local parameters \cite{townsend1976}. One very important universal behavior is the logarithmic dependence of the mean velocity profile in the so-called logarithmic region,

\begin{equation}
{\overline u^+}=\frac{1}{\kappa}\log(y^+)+C_1, \label{eq:lawofwall}
\end{equation}

\noindent  where an overline denotes a mean quantity, the superscript $^+$ denotes normalization by the friction velocity $u_\tau = \sqrt{\tau_w / \rho}$ and kinematic viscosity $\nu$, where $\tau_w$ is the shear stress at the wall and $\rho$ the fluid density; $1/\kappa$ (typically $\kappa$ is called the von K\'{a}rm\'{a}n coefficient) is the slope; $y^+$ is the wall-normal coordinate; and $C_1$ is the intercept at $y^+ = 1$. Equation~\ref{eq:lawofwall} is referred to as the logarithmic law of the wall with constants (at sufficiently high Reynolds number) $\kappa\approx0.4$ and $C_1\approx5$, varying for a given canonical flow type \cite{Nagib2008}. For the distribution of temperature in the boundary layer, using similar dimensional scaling arguments used to derive Eq.~\ref{eq:lawofwall}, yields the law of the wall for the mean temperature distribution,

\begin{equation}
{\overline \theta^+ } =  \overline{\frac{T_w - T(y)}{T_\tau}} = \frac{1}{\kappa_T}\log(y^+)+C_2(Pr), \label{eq:lawofwallT}
\end{equation}

\noindent where $T_w$ is the wall temperature, $T(y)$ is the temperature in the boundary layer, the superscript $^+$ denotes normalization by the friction temperature $T_\tau = q''_w / (\rho c_p u_\tau)$, where $q''_w$ is wall heat flux, and $c_p$ is specific heat. The dimensionless coefficients $1/\kappa_T$ is the slope and $C_2$, which is a function of Prandtl number $Pr$, is the intercept at $y^+=1$. In general, $\kappa_T \approx 0.48$ is taken as a universal constant.  

In non-equilibrium boundary layers, the time scales over which the mean field vary are comparable (or smaller) compared to local turbulent time scales, and the flow field cannot be characterized solely in terms of local parameters \cite{townsend1976}. Such rapid changes in the mean momentum field typically result from pressure gradients, wall curvature, strong three-dimensionality, wall roughness, or dynamic walls. A major focus of non-equilibrium boundary layer flow research is to understand the redistribution of the velocity field when equilibrium is disturbed \cite{Antonia1977, Bradshaw1972a, Bandyopadhyay1993, Castro1998}. In general, these studies show that: (a) in a small local region of a strong perturbation the log-layer is obliterated (i.e.,  the law of the wall given by Eq.~\ref{eq:lawofwall} does not hold), (b) downstream of the perturbation, in the so-called recovery region, an internal stress equilibrium layer grows and the boundary layer recovers towards equilibrium. Conceptually, the effect of (b) relative to Eq.~\ref{eq:lawofwall} is a spatially developing slope and intercept that approach their universal values at the edge of the recovery region; the functional form of the spatial dependence depending on the perturbation. 

While the effects of non-equilibrium on the velocity field have been reasonable well-studied, heat transfer in non-equilibrium boundary layers has received far less attention. \emph{Nevertheless, despite somewhat limited data, it is a well-accepted fact that the law of the wall for temperature is more affected by mean field variations than the velocity field \cite{Blackwell_1972, Kader1991, Bradshaw1995, Kong2001, Houra2006, Wang2008}}. For example, in non-equilibrium boundary layer flow subjected to a pressure gradient, the constants in Eq.~\ref{eq:lawofwallT} vary significantly with pressure gradient while the constants in Eq.~\ref{eq:lawofwall} vary little. This difference in sensitivity is unexpected given that Eqs.~\ref{eq:lawofwall}-\ref{eq:lawofwallT} were derived from analogous dimensional scaling arguments, and has brought into question the validity of the law of the wall \cite{Bradshaw1995}. Moreover, the high sensitivity of the temperature field to pressure gradient flows is remarkable since the pressure gradient does not appear in the transport equation for temperature. The consensus, although not entirely well-understood, is that while the law of the wall for velocity is fairly resilient, the law of the wall for temperature is very strongly affected by mean-field disturbances. The implication is that the scaling used to derive the law of the wall fails to describe the behaviors of the mean dynamics (especially for temperature) when there are large gradients in the mean-field. Consequently, to capture non-equilibrium effects on mean field dynamics, the present state of the research is to reformulate the law of the wall through the use of new scaling laws \cite{Durbin1992, George1993, Cruz1998, Cruz2002, Wang2008} or use single-point closure models (i.e., eddy viscosity or mixing length models) informed by experimental or numerical data \cite{Cebeci1988} %(find ref for Tardet 2009).

The aim of the present work is to study heat transfer in various non-equilibrium boundary layer flow to better understand the effects of non-equilibrium on the temperature distribution and wall heat flux. A major thrust of this work was to develop a unique flow facility to produce and study non-equilibrium and thermal boundary layers. The key design component is a thermal wall plate that can be used to control the thermal boundary conditions in non-equilibrium boundary layer flows. Owing to the strong spatial variations of the flow associated with non-equilibrium boundary layers, this is a nontrivial but important objective needed to better understand thermal transport in these flows. The facility will be used to advance the fundamental knowledge base of non-equilibrium boundary layer transport as well as to develop and validate Reynolds-Averaged-Naiver Stokes (RANS) turbulence models of thermal transport in these flows. \emph{For the latter use, the ability to modify the thermal boundary conditions in a variety of ways is beneficial so that RANS models can be tested and validated in uniquely different flows but in the same flow facility using the same measurement techniques.}  The need for the facility and corresponding measurements is evident by the fact that the experimental measurement of Perry \textit{et al.}\cite{Perry1966} and Blackwell \textit{et al.}\cite{Blackwell_1972} from the late 1960s and early 1970s, respectively, remain the primary datasets utilized in the validation of DNS of thermal boundary layers \cite{Araya2012}. 
