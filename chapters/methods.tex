\chapter{Experimental Methods}

%%%%%%%%%%%%%%%%%%%%%%%%%%%%%%%%%%
\section{Velocity Measurements}
-describe and detail measurements\\


%%%%%%%%%%%%%%%%%%%%%%%%%%%%%%%%%%
\section{Thermal Measurements}
-describe and detail measurements\\

%%%%%%%%%%%%%%%%%%%%%%%%%%%%%%%%%%%%%%%%%
-Methods section from validation paper\\
The temperature of the thermal wall plate and the spatial uniformity of temperature are evaluated using a FLIR SC645 IR camera and J-type thermocouples embedded 2.5mm beneath the top surface of each aluminum plate. The spectral range of the camera is 7.5-13 $\mu m$. The camera resolution is 640 pixel $\times$ 480 pixel and the measurement uncertainty is $\pm$ 2\% of the measured temperature. The measurement uncertainty of the J-type thermocouples is 0.01$^o$C as determined from examining the standard deviation of long time series collected at fixed temperatures. 

The wall-normal profile of temperature within the thermal boundary layer is performed using a sting-mounted type J fine-wire thermocouple with a probe diameter of $0.65mm$ attached to a Velmex BiSlide traverse with a step resolution of 5 $\mu m$.  A Titan Tool Supply Co. Titan Measuring A-1 Microscope-Telescope cathetometer, with an accuracy of $\pm$ 64 $\mu m$ was used to locate the thermocouple position relative to the surface of the thermal wall plate. 
PIV was used to acquire planar fields of velocity in the streamwise/wall-normal plane (i.e., xy-plane in the chosen
experimental coordinate system). %A schematic of the present experimental setup is shown in Fig. xxxx.  
Light is provided by a Photonics DM-series dual cavity Nd:YLF laser capable of 30mJ per pulse. A periscope and a 90$^o$ turning mirror is used to direct the laser light into the tunnel test section. Sheet forming optics (cylindrical and spherical lenses) placed prior to the turning mirror are used to form a laser sheet on the order 1mm thick. The laser light is scattered by atomized oil droplets of 1$\mu m$ nominal diameter introduced into the flow in the seeding manifold upstream of the test-section inlet. Images of the laser light scattered off the tracer particles perpendicular to the incident laser sheets are acquired using two 12-bit Photron FASTCAM SA4 CMOS cameras. 

The CMOS array size of each camera is 1024 pixel $\times$ 1024 pixel. The two cameras are placed on opposite sides of the tunnel and image the same plane but with different field-of-views (FOV). This is done to achieve high spatial resolution in the near wall region while still imaging the entire boundary layer. Camera 1 has a FOV ranging from 2mm $< y <$ 26mm and camera 2 has a FOV ranging from 5mm $< y <$ 54mm, where $y$=0 is the bottom wall. 

The PIV images were acquired at $3.6 kHz$ with a spatial resolution of $0.4\mu m$ per pixel and analyzed using LaVision PIV software, DaVis 8.0.6 and DaVis 8.3.1. Cross-correlation algorithms between two successive images are used to determine the particle displacement field and the time separation between the images is used to determine the velocity field from the displacement field.
 